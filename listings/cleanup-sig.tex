\begin{figure}[ht]
  \caption{Signature of the \texttt{cleanup} function of Irmin databases.}
  \label{lst:cleanup-sig}

  \centering
  \vspace{-1em}
  \begin{minted}{ocaml}
val cleanup :
  ?roots:[ `Branches | `List of commit list ] ->
  ?limit_count:int ->
  ?limit_time:float ->
  t ->
  (unit, [> `Run_again ]) result Lwt.t
(** [cleanup ~roots t] runs the garbage collector on the object graph of
    database [t]. All the commits, nodes and blobs which are not reachable
    from [roots] will be deleted from their back-end stores.

    The collector is concurrent, meaning that collection can be interleaved
    with other database operations. This is done via two mechanisms:
    - The Lwt thread of the collector will yield on every I/O operation to
      allow other threads to execute while waiting for results;
    - The [?limit_count] and [?limit_time] options force the collector to
      return after processing more than a limit number of objects or after
      running for a limit number of seconds. In this case, the function will
      return [Error `Run_again], and it will need to be run again at a later
      time to finish collection. *)
    \end{minted}
\end{figure}
