\begin{figure}[ht]
  \caption{A few basic operations on an Irmin database.}
  \label{lst:irmin-example}

  \centering
  \vspace{-1em}
  \begin{minted}{ocaml}
(* The return values of most operations are wrapped inside the [Lwt.t]
   concurrency monad, so we unwrap them using the OCaml monadic syntax. *)
let (let*) = Lwt.bind

(* Create a new Irmin database with SHA256 hashes, in-memory storage,
   string branches, string paths and floating-point blobs. *)
module Database = Make (Hash.SHA256) (Backend.Memory)
                       (Types.String) (Types.String) (Types.Float)

(* Create the database by forwarding options to the backend. *)
let* t = Database.create () in

(* Checkout the master branch. *)
let* master = Database.checkout t "master" in

(* Commit a first value under path /foo. *)
let* c1 = Database.set ~message:"First commit." master ["foo"] 0.0 in

(* Create a new `checkpoint` branch pointing to the first commit. *)
let* () = Database.Branches.set t "checkpoint" c1 in

(* Commit multiple values at once under paths /bar/baz and /bar/boz. *)
let* c2 = Database.set_tree ~message:"Second commit." master ["bar"]
            (`Node [("baz", `Blob 1.0); ("boz", `Blob 2.0)]) in

(* Retrieve the value of at /bar/baz from the `master` branch.
   Note that this value wouldn't exist in the `checkpoint` branch. *)
let* v = Database.get master ["bar"; "baz"] in assert (v = 0.0)
    \end{minted}
\end{figure}
